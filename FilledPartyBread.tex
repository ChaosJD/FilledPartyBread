  \documentclass[twoside=false, %  doppelseitiger Druck
    DIV=15,% DIV Faktor für Satzspiegelberechnung, sie Doku zu KOMA Script
    BCOR=15mm, % Bindekorrektur
    chapterprefix=false,
    headinclude=true,
    footinclude=false,
    pagesize,%         write pagesize to DVI or PDF
    fontsize=11pt,%             use this font size
    paper=a4,%          use ISO A4
    bibliography=totoc,%         write bibliography-chapter to table of contents
    index=totoc,%         write index-chapter to table of contents
    cleardoublepage=plain,% \cleardoublepage generates pages with pagestyle empty
     headings=big,%       A4/B5
    listof=flat,%        improved list of tables
    numbers=noenddot
  ]{scrbook}

\usepackage[utf8]{inputenc}
\usepackage{makeidx}
\usepackage{amsfonts}
\usepackage[slantedGreek,sc]{mathpazo}  % Schriftart Palatino
% \usepackage{lmodern}    % statt mathpazo, falls CM Fonts verwendet werden sollen
%\usepackage{mathptmx}    % statt mathpazo, falls Times  verwendet werden soll
\usepackage[scaled=.95]{helvet}
\usepackage{courier}
\usepackage[T1]{fontenc}
\usepackage{textcomp}
\usepackage{amsmath}            % standard math notation (vectors/sets/...)
\usepackage{bm}        % standard math notation (fonts)
\usepackage{fixmath}        % standard math notation (fonts)
\usepackage{graphicx}
\usepackage[facing=yes]{floatrow}       % mehrere Gleitobjekte nebeneinander/caption neben Bild/Tabelle
\usepackage[labelfont=bf,sf,font=small,labelsep=space,format=plain]{caption}
\usepackage{subcaption}
\usepackage{scrlayer-scrpage}
% \usepackage{pstool}  % einbinden falls psfrag verwendet werden soll
\usepackage{epstopdf}
\usepackage[ngerman]{babel}
\usepackage{ellipsis}  % Korrigiert den Weißraum um Auslassungspunkte
\usepackage{microtype}  % optischer Randausgleich etc.

\usepackage{xcolor}         % z.B. für schattierte Boxen
\usepackage{framed}			% shaded Umgebung
\definecolor{shadecolor}{gray}{.85}%

% Links im PDF
\usepackage[colorlinks=false,
            pdfborder={0 0 0},
            breaklinks=true]
            {hyperref}


%\typearea[current]{calc}


% Einstellungen für Bild-/Tabellenbeschriftung neben dem Bild
\floatsetup[figure]{capbesideposition={inside,top}}
\floatsetup[table]{capbesideposition={inside,top},style=plaintop}
\renewfloatcommand{fcapside}{figure}[\capbeside][\FBwidth]
\newfloatcommand{tcapside}{table}[\capbeside][\FBwidth]


\selectlanguage{ngerman}


\deffootnote{1em}{1em}{%
 \makebox[1em][l]{\thefootnotemark}}

\makeindex

\newcommand{\real}{\mathord{\mathrm{I\!R}}}

\begin{document}
\selectlanguage{ngerman}
\def\figdir{figures}
\def\tabledir{tables}

\frontmatter

\pagestyle{scrplain}
\pagestyle{empty}

\begin{titlepage}

\sffamily

\raggedleft

\vspace*{-2cm}


\vfill

\centering
\LARGE
% \vspace*{\fill}
%-----------

\Large


\vspace{2cm}

\LARGE

Filled Party Bread 

\vspace{2cm}

\Large

\vspace{1.5cm}


\Large


\vspace{0.5cm}

%\vspace*{\fill}

\LARGE
JD \vspace{1cm}

\vspace{1cm}

\flushleft
 \Large
\vspace*{\fill}

%-----------
\begin{tabbing}
\end{tabbing}
%-----------
Rosenheim, den \today
\end{titlepage}

\cleardoubleemptypage


%%% Local Variables: 
%%% mode: latex
%%% TeX-master: "d"
%%% End: 

\cleardoubleemptypage
\cleardoubleemptypage

%\pagestyle{scrplain}
\pagenumbering{roman}
% ---------------------------------------------------
% D-TOC.TEX zur Verwendung mit TEXPART
% (an eigene Gegebenheiten anzupassen)
% ---------------------------------------------------
%
\tableofcontents
\clearpage
%\listoffigures
\clearpage
%\listoftables
\cleardoublepage


%\pagestyle{scrheadings}


\addtokomafont{caption}{\small}

\mainmatter
\chapter{German/Deutsch}
\section{Zutaten}
\begin{tabular}{c | c}
	500g & Mehl \\
	1 &  Packung Trockenhefe \\
	300-350 ml & (heißes) Wasser \\
	1 & Teelöffel Salz \\
	1 & Eidotter, zum bestreichen\\
\end{tabular}

\section{Füllung}
\begin{tabular}{ c  | c}
	1 & klein geschnittene Zwiebel\\
	600g & geschnittene Champignons \\
	2 & geschnittene Pertersilie \\
	300g & Knoblauchzehen \\
	\textbf{mit Fleisch} & \textbf{mit Fleisch}\\
	300g & gekochter Schinken \\
	\textbf{ohne Fleisch} & \textbf{ohne Fleisch}\\
	300g & geriebener Emmentaler Käse \\
\end{tabular}


\section{Zubereitung Teig}
Aus Mehl, Trockenhefe,
Salz und Wasser Hefeteig herstellen,
kneten und gehen lassen.

\section{Zubereitung Füllung}
besser am Vortag zubereiten

\begin{itemize}
	\item Zwiebeln glasig in Butter dünsten(2 Eßl.)
	\item Champignons, Petersilie, Knoblauch, Schinken zugeben
	\item Würzen mit Salz(wenig, prise)
	\item Mit Pfeffer \& Thymian pikant abschmecken
	\item Erkalten bzw. kalt werden lassen.
\end{itemize}

Teig halbieren, beide Hälften ca. einen halben cm dick ausrollen.
Auf die Teigplatten die  \textbf{kalte} Füllung verteilen.
Teigrand mit Eiklar bestreichen und zusammen rollen.
Auf ein mit Backpapier ausgelegtes Backblech legen.

\begin{flushleft}
	Die Teigrollen nochmals ca. 30 Min. gehen lassen.
	Im Anschluss dünn mit Eidotter bestreichen.
	Bei Ober-  \& Unterhitze im Backrohr (mittleres Fach?)
	ca. 40 Min bei 180°C backen.
\end{flushleft}

\section{Zubereitung vegetarisch}
Die Füllung  \textbf{ohne} Schicken auf den Teig geben. 300g Käse
darüber stzreuen und zusammenrollen.
\chapter{English/Englisch}
\section{Ingredients}
  \chapter{Ingredients}
\chapter{Ingredients}
\input{\tabledir/Ingredients.tex}
%%%%%%%%%%%%%%%%%%%%%%%%%%%%%%%%%%%%%%%%%%%%%%%%%%%%%%%%%%%%%%

%%%%%%%%%%%%%%%%%%%%%%%%%%%%%%%%%%%%%%%%%%%%%%%%%%%%%%%%%%%%%%


\section{Filling}
\begin{tabular}{c | c}
  1 & big chopped onion\\
  600g & cutted champions \\
  2 & bundles chopped parsley \\
  300g & Garlic cloves \\
  \textbf{with meat} & \textbf{with meat}\\
  300g & cooked diced ham \\
  \textbf{without meat} & \textbf{without meat}\\
  300g & grated emmental cheese \\
\end{tabular}

\section{Preparation of Dough}
Made from flour, dry yeast,
salt and water to make yeast dough,
knead and let rise.

\section{Preparation of the filling, it is better to prepare it the day before}
\begin{enumerate}
	\item{Sauté onions in butter (2 tbsp.)}
	\item{Add mushrooms, parsley, garlic, ham}
	\item{Season with salt (little, pinch)}
	\item{Season to taste with pepper \& thyme}
	\item{Allow to cool or become cold.}
\end{enumerate}
Halve the dough, roll out both halves to about half a cm thick.
Spread the \textbf{cold} filling on the dough sheets.
Brush the edges of the dough with egg white and roll up.
Place on a baking tray covered with baking paper.
\begin{flushleft}
	Let the rolls rise for another 30 minutes.
	Then brush lightly with egg yolk.
	With top and bottom heat in the oven (middle compartment?)
	Bake for approx. 40 minutes at 180°C.
\end{flushleft}

\section{Preparation Vegetarian}
Put the filling \textbf{without} sending on the dough. 300g cheese
spread over and roll up.
%\chapter{Ingredients}
\chapter{Ingredients}
\chapter{Ingredients}
\input{\tabledir/Ingredients.tex}
%%%%%%%%%%%%%%%%%%%%%%%%%%%%%%%%%%%%%%%%%%%%%%%%%%%%%%%%%%%%%%

%%%%%%%%%%%%%%%%%%%%%%%%%%%%%%%%%%%%%%%%%%%%%%%%%%%%%%%%%%%%%%

%%%%%%%%%%%%%%%%%%%%%%%%%%%%%%%%%%%%%%%%%%%%%%%%%%%%%%%%%%%%%%

%\begin{tabular}{c | c}
  1 & big chopped onion\\
  600g & cutted champions \\
  2 & bundles chopped parsley \\
  300g & Garlic cloves \\
  \textbf{with meat} & \textbf{with meat}\\
  300g & cooked diced ham \\
  \textbf{without meat} & \textbf{without meat}\\
  300g & grated emmental cheese \\
\end{tabular}
%\input{\tabledir/PreparationDough}
%\chapter{Preparation of the filling, it is better to prepare it the day before}
\begin{enumerate}
  \item{Sauté onions in butter (2 tbsp.)}
  \item{Add mushrooms, parsley, garlic, ham}
  \item{Season with salt (little, pinch)}
  \item{Season to taste with pepper \& thyme}
  \item{Allow to cool or become cold.}
\end{enumerate}
Halve the dough, roll out both halves to about half a cm thick.
Spread the \textbf{cold} filling on the dough sheets.
Brush the edges of the dough with egg white and roll up.
Place on a baking tray covered with baking paper.
\begin{flushleft}
Let the rolls rise for another 30 minutes.
Then brush lightly with egg yolk.
With top and bottom heat in the oven (middle compartment?)
Bake for approx. 40 minutes at 180°C.
\end{flushleft}

%\chapter{Preparation Vegetarian}
Put the filling \textbf{without} sending on the dough. 300g cheese
spread over and roll up.

\appendix

\cleardoublepage

\bibliographystyle{natger}
%\bibliography{thesis} % Literarturverzeichnis

\cleardoublepage


\footnotesize
\printindex


\end{document}
