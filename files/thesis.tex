  \documentclass[twoside=false, %  doppelseitiger Druck
    DIV=15,% DIV Faktor für Satzspiegelberechnung, sie Doku zu KOMA Script
    BCOR=15mm, % Bindekorrektur
    chapterprefix=false,
    headinclude=true,
    footinclude=false,
    pagesize,%         write pagesize to DVI or PDF
    fontsize=11pt,%             use this font size
    paper=a4,%          use ISO A4
    bibliography=totoc,%         write bibliography-chapter to table of contents
    index=totoc,%         write index-chapter to table of contents
    cleardoublepage=plain,% \cleardoublepage generates pages with pagestyle empty
     headings=big,%       A4/B5
    listof=flat,%        improved list of tables
    numbers=noenddot
  ]{scrbook}

\usepackage[utf8]{inputenc}
\usepackage{makeidx}
\usepackage{amsfonts}
\usepackage[slantedGreek,sc]{mathpazo}  % Schriftart Palatino
% \usepackage{lmodern}    % statt mathpazo, falls CM Fonts verwendet werden sollen
%\usepackage{mathptmx}    % statt mathpazo, falls Times  verwendet werden soll
\usepackage[scaled=.95]{helvet}
\usepackage{courier}
\usepackage[T1]{fontenc}
\usepackage{textcomp}
\usepackage{amsmath}            % standard math notation (vectors/sets/...)
\usepackage{bm}        % standard math notation (fonts)
\usepackage{fixmath}        % standard math notation (fonts)
\usepackage{graphicx}
\usepackage[facing=yes]{floatrow}       % mehrere Gleitobjekte nebeneinander/caption neben Bild/Tabelle
\usepackage[labelfont=bf,sf,font=small,labelsep=space,format=plain]{caption}
\usepackage{subcaption}
\usepackage{scrlayer-scrpage}
% \usepackage{pstool}  % einbinden falls psfrag verwendet werden soll
\usepackage{epstopdf}
\usepackage[ngerman]{babel}
\usepackage{ellipsis}  % Korrigiert den Weißraum um Auslassungspunkte
\usepackage{microtype}  % optischer Randausgleich etc.

\usepackage{xcolor}         % z.B. für schattierte Boxen
\usepackage{framed}			% shaded Umgebung
\definecolor{shadecolor}{gray}{.85}%

% Links im PDF
\usepackage[colorlinks=false,
            pdfborder={0 0 0},
            breaklinks=true]
            {hyperref}


%\typearea[current]{calc}


% Einstellungen für Bild-/Tabellenbeschriftung neben dem Bild
\floatsetup[figure]{capbesideposition={inside,top}}
\floatsetup[table]{capbesideposition={inside,top},style=plaintop}
\renewfloatcommand{fcapside}{figure}[\capbeside][\FBwidth]
\newfloatcommand{tcapside}{table}[\capbeside][\FBwidth]


\selectlanguage{ngerman}


\deffootnote{1em}{1em}{%
 \makebox[1em][l]{\thefootnotemark}}

\makeindex

\newcommand{\real}{\mathord{\mathrm{I\!R}}}

\begin{document}
\selectlanguage{ngerman}
\def\figdir{figures}
\def\tabledir{tables}

\frontmatter

\pagestyle{scrplain}
\pagestyle{empty}

\begin{titlepage}

\sffamily

\raggedleft

\vspace*{-2cm}


\vfill

\centering
\LARGE
% \vspace*{\fill}
%-----------

\Large


\vspace{2cm}

\LARGE

Filled Party Bread 

\vspace{2cm}

\Large

\vspace{1.5cm}


\Large


\vspace{0.5cm}

%\vspace*{\fill}

\LARGE
JD \vspace{1cm}

\vspace{1cm}

\flushleft
 \Large
\vspace*{\fill}

%-----------
\begin{tabbing}
\end{tabbing}
%-----------
Rosenheim, den \today
\end{titlepage}

\cleardoubleemptypage


%%% Local Variables: 
%%% mode: latex
%%% TeX-master: "d"
%%% End: 

\cleardoubleemptypage
\cleardoubleemptypage

\pagestyle{scrplain}
\pagenumbering{roman}
% ---------------------------------------------------
% D-TOC.TEX zur Verwendung mit TEXPART
% (an eigene Gegebenheiten anzupassen)
% ---------------------------------------------------
%
\tableofcontents
\clearpage
%\listoffigures
\clearpage
%\listoftables
\cleardoublepage


\pagestyle{scrheadings}


\addtokomafont{caption}{\small}

\mainmatter

\chapter{Ingredients}
\chapter{Ingredients}
\chapter{Ingredients}
\input{\tabledir/Ingredients.tex}
%%%%%%%%%%%%%%%%%%%%%%%%%%%%%%%%%%%%%%%%%%%%%%%%%%%%%%%%%%%%%%

%%%%%%%%%%%%%%%%%%%%%%%%%%%%%%%%%%%%%%%%%%%%%%%%%%%%%%%%%%%%%%

%%%%%%%%%%%%%%%%%%%%%%%%%%%%%%%%%%%%%%%%%%%%%%%%%%%%%%%%%%%%%%

\begin{tabular}{c | c}
  1 & big chopped onion\\
  600g & cutted champions \\
  2 & bundles chopped parsley \\
  300g & Garlic cloves \\
  \textbf{with meat} & \textbf{with meat}\\
  300g & cooked diced ham \\
  \textbf{without meat} & \textbf{without meat}\\
  300g & grated emmental cheese \\
\end{tabular}


\appendix

\cleardoublepage

\bibliographystyle{natger}
\bibliography{thesis}

\cleardoublepage


\footnotesize
\printindex


\end{document}
